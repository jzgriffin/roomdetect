\documentclass{article}

\title{Report: Room Detector}
\author{Jeremiah Griffin}
\date{Spring 2017}

\begin{document}

\pagenumbering{gobble}
\maketitle
\newpage

\pagenumbering{roman}
\tableofcontents
\listoffigures
\listoftables
\newpage

\pagenumbering{arabic}

\section{Introduction}

The room detector uses inputs from several sensors to determine what
room or environment the device is currently in.  It uses a statistical
model to associate ambient factors with a user-defined room number.  The
network must be trained by the user before it may be used.  This is done
online by placing the device in each room to be detected, entering
training mode, associating either a new or existing room number with the
training data, and then moving the device through the room.  The longer
the device is trained, the more accurate its classifications become.
Once training is complete, the device may enter detection mode and be
moved between any of the trained rooms.  If brought to an untrained
room, it will make its best guess at classifying the room as one that
was trained.

% TODO: Insert photo of project

\section{Hardware}

\subsection{Parts}

The hardware components that were used in this project are listed below.
The equipment that was not taught as part of the course material is
listed in bold.

\begin{table}[h!]
  \centering
  \begin{tabular}{llr}
    Part Number & Component & Quantity \\
    \hline
    ATmega1284p             & Microcontroller                  & 2 \\
                            & SPDT toggle switch               & 1 \\
                            & SPST push-down button            & 3 \\
                            & 10K\Omega\ trimmer potentiometer & 1 \\
                            & 330\Omega\ 10\% resistor         & 4 \\
                            & CdS photoresistor                & 1 \\
    LCM-S01602DTR/M         & Liquid crystal display           & 1 \\
    \textbf{TMP36}          & Temperature sensor               & 1 \\
    \textbf{CMA-4544PF-W}   & Electret microphone              & 1 \\
    \textbf{MAX9814}        & Microphone amplifier             & 1
  \end{tabular}
  \caption{Listing of parts used in design}
  \label{table:1}
\end{table}

\subsection{Pinout}

\section{Software}

The software designed for this project was implemented using the PES
standard.  The overall design as a task diagram is included below.

% TODO: Insert task diagram

In the following sections, the individual tasks from the above diagram
are briefly described in addition to their inputs and outputs.  The
state machines for these tasks are presented in appendix \ref{app.sm}.

\subsection{Tasks}

\section{Complexities}

\subsection{Complete}

\begin{itemize}

  \item Using EEPROM to persist the classification model state

  \item Using USART to offload computation to a dedicated
    microcontroller

  \item Using ADC multiplexing to collect analog readings from multiple
    sensors

  \item Using a temperature sensor and microphone obtained outside of
    the lab kit

  \item Using a statistical model to perform input classification using
    fixed-point arithmetic

\end{itemize}

\subsection{Incomplete}

\begin{itemize}

  \item Using a neural network to perform input classification using
    fixed-point arithmetic
    % TODO: Explain why this was not completed

\end{itemize}

\section{Links}

\begin{description}

  %\item [Demonstration Video] https://youtu.be/%TODO

  \item [GitHub Repository] https://github.com/nokurn/roomdetect

\end{description}

\section{Known Issues}

\begin{itemize}

  \item The classification algorithm does not cope well with noise and
    may confuse environments with a large degree of noise as being
    equivalent regardless of the specifics of that noise.

  \item The temperature sensor is not measured with sufficient precision
    for its readings to adequately influence the results of the
    classification.

  \item The microphone primarily distinguishes between noisy and quiet
    environments with no consideration of the qualities of the sound
    such as timbre and pitch.

  \item The classifier most heavily relies on the measurements from the
    photo-resistor, despite unbiased treatment of all readings in the
    input vector.  This is likely due to the aforementioned reasons:
    low analog precision and the need for pre-processing of some
    readings.

\end{itemize}

\section{Future Work}

\begin{itemize}

  \item The largest improvement might come from replacing the compute
    microcontroller with a higher-power compute unit, possibly
    integrating floating-point arithmetic, and using a more robust
    classification model such as a neural network.

  \item By using an external analog-digital converter with higher
    precision, or simply replacing the ATmega with a microcontroller
    with a higher-precision integrated ADC, sensors with small output
    ranges would contribute more to the accuracy and confidence of the
    device's classifications.  This would reflect immediately in the
    ability of the device to distinguish between environments with
    slightly different temperatures.

  \item Applying a Fourier transform to the microphone signal and using
    the individual components as input to the classifier would improve
    the quality of classifications with regard to environmental sounds.
    For example, if one room contains several computers with spinning
    fans and another room has a single ceiling fan, yet the volume level
    in both rooms is similar, a Fourier analysis of the sound waves
    would enable the classifier to distinguish between the two by the
    timbre and pitch of the sound in addition to its loudness.

  \item Adding additional sensors would increase the size of the
    classifier's feature vector and its accuracy.  Barometric pressure,
    humidity, altitude, or even GPS would be likely candidates for
    inclusion in the system.

\end{itemize}

\appendix

\section{Appendix: State Machines}
\label{app.sm}

\subsection{Master Microcontroller}
\label{app.sm.master}

\subsection{Compute Microcontroller}
\label{app.sm.compute}

\subsection{Common Infrastructure}
\label{app.sm.common}

\end{document}
