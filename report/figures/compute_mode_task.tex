\begin{flushleft}
  \begin{description}
    \item [Input]
      \verb#usart_packet_rx_queue#
    \item [Output]
      \verb#device_mode: enum#;
      \verb#training_room: uint8#
    \item [Period] 16 ms
  \end{description}
\end{flushleft}

\iftoggle{TASKSPECONLY}{}
{
\begin{center}
  \begin{tikzpicture}[->,>=stealth',
      shorten >=1pt,
      node distance=41mm,
      auto,
      on grid,
      semithick]

    \node[state,initial]
      (Initial) {Initial};
    \node[state]
      (Configure) [right=of Initial] {Configure};
    \node[state]
      (Ready) [right=of Configure] {Ready};
    \node[state]
      (CheckPacket) [below of=Ready] {CheckPacket};
    \node[state]
      (DiscardPacket) [below of=Initial] {DiscardPacket};
    \node[state]
      (ReceiveDetection) [below of=CheckPacket] {ReceiveDetection};
    \node[state]
      (ReceiveTraining) [below of=DiscardPacket] {ReceiveTraining};

    \path[->]
      (Initial) edge node {} (Configure)
      (Configure) edge node {} (Ready)
      (Ready) edge [align=left] node {!is\_usart\_\\packet\_rx\_\\queue\_empty()} (CheckPacket)
      (Ready) edge [loop right] node {} (Ready)
      (CheckPacket) edge node {!is\_packet\_good()} (DiscardPacket)
      (CheckPacket) edge node {has\_detection\_packet()} (ReceiveDetection)
      (CheckPacket) edge node {has\_training\_packet()} (ReceiveTraining);

  \end{tikzpicture}
\end{center}

The following transitions have been omitted for clarity: from the
DiscardPacket, ReceiveDetection, and ReceiveTraining states to the Ready
state under any condition.

}
